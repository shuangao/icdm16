In this section, we focus on the Phrase I of HTL and introduce our biased regularization for binary LS-SVM for our problem. We propose a new perspective for transfer learning in HTL and analysis the reasons why negative transfer could happen.

\subsection{Feature augmentation in HTL}
We define our transfer task in the following way: Suppose we have $N$ visual categories. 
In our source task, $N$ source binary classifiers $f'_n(x)$ for $n=1,...,N$, are trained from a distribution $\mathcal{D}_s$ to distinguish whether an object belongs to each of the $N$ categories. In our target task, we have another small set of data $(x,y)$ drawn from another distribution $\mathcal{D}_t$ with the same $N$ categories as those in source task. We want to train $N$ target binary classifiers $f_n(x)$ for $n=1,...,N$ on the data of the target domain so that they can perform well on the target domain.

Vapnik et al. \cite{vapnik2015learning} proposed a diagram to transfer the privileged knowledge from the teacher to the student. The privileged knowledge is used as the auxiliary feature for training the student model. Inspired from this idea, we propose our feature augmentation approach for  HTL in domain adaptation. We treat the outputs of the source models as the auxiliary features and used them to train the target model. The difference between the auxiliary feature in our work and the privileged information is that we can use the auxiliary feature in both training and testing procedures while privileged knowledge can be used for training the model only.

\begin{figure*}
	\centering
	\includegraphics[scale=0.5]{fig/aug2.png}
	\caption{The transfer learning process can be considered as the augmentation of the target data where the decision scores of the source models are appended as the auxiliary features. The transfer parameters can be considered as the a part of the corresponding hyperplane. We can consider 2 augmentation strategies: multi-adaptation and single-adaptation.}\label{fig:aug}
\end{figure*}

As a result, when there are a group of $N$ source models $F'(x)=\{f'_i(x)|i=1,...,n\}$, we can augment the target feature space as  $\hat{x}=[\phi(x),f'_1(x),...,f'_N(x)]$ when we want to leverage the knowledge from multiple source models (Multi-adaptation) or $\hat{x}=[\phi(x),f'_r(x)], r\in 1,...,N$ (Single-adaptation). Here $\phi(x)$ can be any feature mapping that maps the example into another space. Thus the target binary model for category $n$ can be represented as $f_n=\hat{w}_n\hat{x}$, where $\hat{w}_n=[w_{n}'',\beta_1,...,\beta_N]$ (Multi-adaptation) or $[w_{n}'',\beta_r]$ (Single-adaptation). Here, we call the element(s) $\beta$ the \textit{Transfer Parameter}. 

\begin{equation}\label{eq:aug_pre}
\begin{aligned}
f_n(x)&=w_{n}''\phi(x)+\sum\limits_k^N{\beta _kw'_k\phi(x)}
\end{aligned}
\end{equation}
An intuitive interpretation of Eq. \eqref{eq:aug_pre} is that the auxiliary features can be considered as a similarity score from the source model. When the target model decides whether an object belongs to certain category, it also considers the decisions of the source model as the reference (the second part of the right hand side in Eq. \eqref{eq:aug_pre}).

$f_n$ can be formalized as the following optimization problem:
\begin{equation}\label{eq:svm_obj}
\begin{aligned}
\textbf{min} && \Omega(\hat{w}_n) + \frac{C}{2}\sum\limits_i^l {\mathcal{L}(f_n(x_i),Y_{in})} \\
\end{aligned}
\end{equation}
Here $\Omega({w})$ is the regularization term to guarantee good generalization performance and avoid overfitting. $Y_{in}$ is the encoded label for binary classifier following $Y_{in}=1$ if $y_i=n$ and $-1$ otherwise. $\mathcal{L}(\cdot)$ is the loss function. When we consider to use Least Square SVM as the classifier, we have $\mathcal{L}(f,y) = (f-y)^2$ and $\Omega({w}) = \frac{1}{2}||w||^2$. 

Compared to the previous methods in HTL, we have the following advantages:
\begin{itemize}
	\item Wider range selection of the source models. Most of the previous works in HTL are limited to their source models to the linear models \cite{tommasi2014learning} \cite{aytar2011tabula}. With the idea of feature augmentation, we can treat the source model as a whole which just outputs the decision score of an example. Therefore, we can exploit the knowledge of the different types, such as Neural Networks and inference models \footnote{Previous works, such as \cite{tommasi2014learning} \cite{aytar2011tabula}, can be considered as the special case of our problem where the linear model is used as the source. We can obtain similar objective function when we consider the transfer parameter as the hyper-parameter that can should be defined according to the background knowledge.}.
	\item Insight into the performance of the target model. By feature augmentation, we turn the HTL domain adaptation problem into a classical learning problem and we can better analysis some important problems existed in transfer learning such as negative transfer. In the next part, we introduce some analysis of how to improve the performance of the target model in our framework. 
\end{itemize}

\subsection{Reasons for negative transfer}
From the perspective of the feature augmentation, we can turn the problem of domain adaptation problem with HTL into a traditional learning problem, i.e. find the optimal values for the elements of the hyperplane hyperplane $\hat{w}=[w_{n}'',\beta_1,...,\beta_N]$ \footnote{We take Mult-adaptation for instance. The conclusion can be applied to Single-adaptation without any modification.}. According to the principle of Structural Risk Minimization (SRM) \cite{vapnik1999overview}, the risk of a linear classifier $f(x)=wx$ on the unseen test data $R(f)$ (generalization risk) is bounded by:
\begin{equation}\label{eq:srm}
R(f) \le {R_{emp}}(f) + \sqrt {\frac{{h(\ln (2l/h) + 1) + \ln (\delta /4)}}{l}} 
\end{equation}
Here the first part on the right-hand side of the inequation ${R_{emp}}(f)$ is the empirical risk (training error) of the classifier and the second part is the confidence interval. $h$ and $l$ denote the VC dimension and number of training data of the classifier respectively and $\delta$ is the confidential parameter. According to \cite{suykens1999least}, the VC dimension $h$ is bounded by $h \le \min([||w||^2R^2],l)+1$ where $R$ is the radius of the smallest ball containing data $x$ and $||w||$ is the \textit{2-norm} of the hyperplane.

As we discussed above, we use the outputs of the source models as the auxiliary features to augment the target data. Let $R$ and $\hat{R}$ denote the radius of the data before and after augmentation. We should have $R^2 \le \hat{R}^2$ and $||w||^2\le ||\hat{w}||^2$. This indicates that the VC dimension of the target model trained on the augmented data (augmented model) tends to increase compared to the model trained from the original data, i.e.  the method without transferring any source knowledge (no transfer model). As a result, feature augmentation eventually increases the confidence interval of the risk of the augmented model. When the augmented model failed to decrease the empirical risk, its performance would degrade, i.e. suffer from negative transfer. For example, when the auxiliary features can't provide any extra useful information for classification, i.e. the source domain and target domain are unrelated, negative transfer could happen. In contrast, if we can significantly decrease the empirical risk of the augmented model, we can decrease its generalization risk and get improved performance, i.e. positive transfer.

From the analysis above, we can see that in order to get improved performance and alleviate negative transfer, we have to carefully control the value of the transfer parameters. Following this idea, a simple method to estimate the hyperplane $\hat{w}$ for the target model solve optimizing problem \eqref{eq:svm_obj} directly. However, if we are not able to regularize the transfer parameters properly, the target model could easily suffer from negative transfer when the source and target are weakly related (see the experiment result for the method Feature+ in Section \ref{sec:exp}).
In next section, we introduce our method SMTLe to estimate the transfer parameters that can improve the performance of the target model and alleviate negative transfer .

