The motivation of transfer knowledge between different domains is to apply the previous information from the source domain to the target one, assuming that there exists certain relationship, explicit or implicit, between the feature space of these two domains \cite{pan2010survey}. Technically, previous work can be concluded into solving the following three issues: what, how and when to transfer \cite{tommasi2014learning}.


\textbf{What to transfer.} Previous work tried to answer this question from three different aspects: selecting transferable instances, learning transferable feature representations and transferable model parameters. Instance-based transfer learning assumes that part of the instances in the source domain could be re-used to benefit the learning for the target domain. Lim et al. proposed a method of augmenting the training data by borrowing data from other classes for object detection \cite{lim2012transfer}. Learning transferable features means to learn common feature that can alleviate the bias of data distribution in the target domain. Recently, Long et al. proposed a method that can learn transferable features with deep neural network and showed some impressive results on the  benchmarks \cite{LongICML15}. Model transfer
approach assumes that the parameters of the model for the source task can be transferred to the target task. Yang et al. proposed Adaptive SVMs by transferring parameters by incorporating the auxiliary classifier trained from source domain \cite{yang2007cross}. On top of Yang's work, Ayatar et al. proposed PMT-SVM that can determine the transfer regularizer according to the target data automatically \cite{aytar2011tabula}. Tommasi et al. proposed Multi-KT that can utilize the parameters from multiple source models for the target classes  \cite{tommasi2014learning}.
Kuzborskij et al. proposed a similar method to learn new categories by leveraging over the known source \cite{kuzborskij2013n}.

\textbf{When and how to transfer.} The question \textit{when to transfer} arises when we want to know if the information acquired from the previous task is relevant to the new one (i.e. in what situation, knowledge should not be transferred). 
\textit{How to transfer} the prior knowledge effectively should be carefully designed to prevent inefficient and negative transfer. Some previous work consists in using generative probabilistic method \cite{davis2009deep} \cite{wang2014active} \cite{zhou2014multi}.  Bayesian learning methods can predict the target domain by combining the prior source distribution to generate a posterior distribution. Alternatively, some previous max margin methods show that it is possible to learn from a few examples by minimizing the  Leave-One-Out (LOO) error for the training model \cite{kuzborskij2013n} \cite{tommasi2010safety}. Cawley et al. show that there is a closed-form implementation of LOO cross-validation that can generate unbiased model estimation for LS-SVM \cite{cawley2006leave}.

Our work corresponds to the context above. In this paper, we propose SMTLe based on model transfer approach with LS-SVM. We address our work on how to prevent negative transfer while just accessing the source model for domain adaptation. Compared to other works, propose a new perspective on the previous work of HTL, which brings more insight to negative transfer. Then we propose a novel strongly convex objective function for transfer parameters estimation. We show that SMTLe can converge at the rate of $O(\frac{\log(t)}{t})$. 
By optimizing this objective function, SMTLe can autonomously adjust the transfer parameters for different hypotheses. We theoretically show that, without any data distribution assumption, the superior bound of the training loss for SMTLe is the loss of a method learning directly (i.e. without using any prior knowledge). As a result, SMTLe can achieve a better performance and alleviate negative transfer.
